\documentclass[10pt,a4paper,]{article}
\usepackage[]{tgpagella}
\usepackage{amssymb,amsmath}
\usepackage{ifxetex,ifluatex}
\usepackage{fixltx2e} % provides \textsubscript
\ifnum 0\ifxetex 1\fi\ifluatex 1\fi=0 % if pdftex
  \usepackage[T1]{fontenc}
  \usepackage[utf8]{inputenc}
\else % if luatex or xelatex
  \ifxetex
    \usepackage{mathspec}
  \else
    \usepackage{fontspec}
  \fi
  \defaultfontfeatures{Ligatures=TeX,Scale=MatchLowercase}
\fi
% use upquote if available, for straight quotes in verbatim environments
\IfFileExists{upquote.sty}{\usepackage{upquote}}{}
% use microtype if available
\IfFileExists{microtype.sty}{%
\usepackage{microtype}
\UseMicrotypeSet[protrusion]{basicmath} % disable protrusion for tt fonts
}{}
\usepackage[margin=2.25cm]{geometry}
\usepackage{hyperref}
\hypersetup{unicode=true,
            pdftitle={Addressing Subposterior Conflict With Weighted Proposals},
            pdfauthor={Andrew Manderson},
            pdfborder={0 0 0},
            breaklinks=true}
\urlstyle{same}  % don't use monospace font for urls
\usepackage{graphicx,grffile}
\makeatletter
\def\maxwidth{\ifdim\Gin@nat@width>\linewidth\linewidth\else\Gin@nat@width\fi}
\def\maxheight{\ifdim\Gin@nat@height>\textheight\textheight\else\Gin@nat@height\fi}
\makeatother
% Scale images if necessary, so that they will not overflow the page
% margins by default, and it is still possible to overwrite the defaults
% using explicit options in \includegraphics[width, height, ...]{}
\setkeys{Gin}{width=\maxwidth,height=\maxheight,keepaspectratio}
\IfFileExists{parskip.sty}{%
\usepackage{parskip}
}{% else
\setlength{\parindent}{0pt}
\setlength{\parskip}{6pt plus 2pt minus 1pt}
}
\setlength{\emergencystretch}{3em}  % prevent overfull lines
\providecommand{\tightlist}{%
  \setlength{\itemsep}{0pt}\setlength{\parskip}{0pt}}
\setcounter{secnumdepth}{5}
% Redefines (sub)paragraphs to behave more like sections
\ifx\paragraph\undefined\else
\let\oldparagraph\paragraph
\renewcommand{\paragraph}[1]{\oldparagraph{#1}\mbox{}}
\fi
\ifx\subparagraph\undefined\else
\let\oldsubparagraph\subparagraph
\renewcommand{\subparagraph}[1]{\oldsubparagraph{#1}\mbox{}}
\fi

%%% Use protect on footnotes to avoid problems with footnotes in titles
\let\rmarkdownfootnote\footnote%
\def\footnote{\protect\rmarkdownfootnote}

%%% Change title format to be more compact
\usepackage{titling}

% Create subtitle command for use in maketitle
\providecommand{\subtitle}[1]{
  \posttitle{
    \begin{center}\large#1\end{center}
    }
}

\setlength{\droptitle}{-2em}

  \title{Addressing Subposterior Conflict With Weighted Proposals}
    \pretitle{\vspace{\droptitle}\centering\huge}
  \posttitle{\par}
    \author{Andrew Manderson}
    \preauthor{\centering\large\emph}
  \postauthor{\par}
      \predate{\centering\large\emph}
  \postdate{\par}
    \date{06 August, 2019}

\usepackage{amsmath}
% I always seem to need tikz for something
\usepackage{tikz}
\usetikzlibrary{positioning, shapes, intersections, through, backgrounds, fit, decorations.pathmorphing}

\usepackage{setspace}
\onehalfspacing

\usepackage{lineno}
% \linenumbers

% required for landscape pages. beware, they back the build very slow.
\usepackage{pdflscape}

% table - `gt' package uses these, often unimportant
% \usepackage{longtable}
% \usepackage{booktabs}
% \usepackage{caption}

\setcounter{secnumdepth}{3}

% pd stands for: probability distribution and is useful to distringuish
% marignals for probabilities specifically p(p_{1}) and the like.
\newcommand{\pd}{\text{p}}
\newcommand{\q}{\text{q}}
\newcommand{\w}{\text{w}}
\newcommand{\pdr}{\text{r}}
\newcommand{\pdrh}{\hat{\text{r}}}

% melding
\newcommand{\ppoolphi}{\pd_{\text{pool}}(\phi)}
\newcommand{\pmeld}{\pd_{\text{meld}}}

% the q(x)w(x), "weighted target" density 
% for the moment I'm going to call it s(x), as that is the next letter of the 
% alphabet. Can change it later
\newcommand{\s}{\text{s}}
% direct density estimate - replaces lambda.
\newcommand{\ddest}{\text{s}}
% target weighting function
\newcommand{\tarw}{\text{u}}

% constants - usually sizes of things
\newcommand{\Nx}{N}
\newcommand{\Nnu}{\text{N}_{\text{nu}}}
\newcommand{\Nde}{\text{N}_{\text{de}}}
\newcommand{\Nmc}{\text{N}_{\text{mc}}}
\newcommand{\Nw}{W}
\newcommand{\Nm}{M}

% locales - could switch to x and x'
\newcommand{\xnu}{x_{\text{nu}}}
\newcommand{\xde}{x_{\text{de}}}

% sugiyama stuff
\newcommand{\pdnu}{\pd_{\text{nu}}}
\newcommand{\pdde}{\pd_{\text{de}}}

% indices 
\newcommand{\wfindex}{w}
\newcommand{\sampleindex}{n}
\newcommand{\modelindex}{m}

%% operators that need additional spacing
\newcommand{\wpropto}{\,\,\propto\,\,}

\begin{document}
\maketitle

The need for accurate self-density ratio estimates arose from a conflict
between \((m - 1)^{\text{th}}\) stage posterior and the unknown
\(m^{\text{th}}\) model's prior marginal distribution. This was purely a
disagreement of scale, the distributions in question did not have
disjoint supports. This document considers subposteriors with disjoint
supports, and demonstrates some potential solutions.

\section{Disjoint subposteriors}\label{disjoint-subposteriors}

Consider the two subposteriors \(\pd_{1}(\phi \mid Y_{1})\),
\(\pd_{2}(\phi \mid Y_{2})\), and melded posterior
\(\pmeld(\phi \mid Y_{1}, Y_{2})\) shown in
Figure~\ref{fig:subpost_disagreement}. Using the samples of either of
the subposteriors as a proposal for the overall posterior results in a
degenerate sample of the overall posterior, due to an insufficient
number of samples in the support of \(\pmeld(\phi \mid, Y_{1}, Y_{2})\).
If instead we sample a version of one of the subposteriors which is
\emph{augmented} by a function \(\tarw_{1}(\phi; \eta_{1})\),
i.e.~\(\pd_{1}(\phi \mid Y_{1})\tarw_{1}(\phi; \eta_{1})\), we can
control the stage two acceptance probability through
\(\tarw_{1}(\phi; \eta_{1})\). The green line in
Figure~\ref{fig:subpost_disagreement} illustrates how we can suitably
scale and shift the stage one target.

\begin{figure}

{\centering \includegraphics[width=1\linewidth]{plots/intro/subpost-disagreement} 

}

\caption{Hypothetical conflicting subposteriors (blue), melded posterior (red), and the augmented stage one one target (green, dashed).}\label{fig:subpost_disagreement}
\end{figure}

The general problem concerns instances where the support of
\(\pd_{\text{meld}, \modelindex - 1}(\phi, \psi_{1}, \ldots, \psi_{\modelindex - 1} \mid Y_{1}, \ldots, Y_{\modelindex - 1})\)
does not intersect with the support of
\(\pd_{\text{meld}, \modelindex}(\phi, \psi_{1}, \ldots, \psi_{\modelindex} \mid Y_{1}, \ldots, Y_{\modelindex})\),
hence also does not overlap with
\(\pd_{\modelindex}(\phi,~\psi_{\modelindex}~\mid~Y_{\modelindex})\). We
must choose the functional form of
\(\tarw_{\modelindex - 1}(\phi; \eta_{\modelindex - 1})\), as well as
the parameter \(\eta_{\modelindex - 1}\), such that the supports of
\(\pd_{\text{meld}, \modelindex - 1}(\phi, \psi_{1}, \ldots, \psi_{\modelindex - 1} \mid Y_{1}, \ldots, Y_{\modelindex - 1})\tarw_{\modelindex - 1}(\phi; \eta_{\modelindex - 1})\)
and
\(\pd_{\text{meld}, \modelindex}(\phi, \psi_{1}, \ldots, \psi_{\modelindex} \mid Y_{1}, \ldots, Y_{\modelindex})\)
intersect.

\section{Methods - Maths}\label{methods---maths}

Consider the overall melded target posterior for \(\Nm = 2\) models
\begin{equation}
  \pmeld (\phi, \psi_{1}, \psi_{2} \mid Y_{1}, Y_{2}) \wpropto
  \ppoolphi
  \frac{
    \pd_{1}(\phi, \psi_1, Y_{1})
  } {
    \pd_{1}(\phi)
  }
  \frac{
    \pd_{2}(\phi, \psi_2, Y_{2})
  } {
    \pd_{2}(\phi)
  }.
  \label{eqn:melded-target-posterior}
\end{equation} The
\emph{augmented stage one target} is then
\begin{equation}
  \pd_{\text{meld}, 1}(\phi, \psi_{1}, Y_{1}) \wpropto
  \frac {
    \pd_{1}(\phi, \psi_1, Y_{1})
  } {
    \pd_{1}(\phi)
  }
  \tarw_{1}(\phi; \eta_{1})
  =
  \mathcal{Q}_{1}(\phi).
  \label{eqn:stage-one-target}
\end{equation} Hence, the stage two
acceptance probability is
\begin{align}
 \alpha(\phi^{*}, \phi) 
  &= 
    \frac {
      \pmeld(\phi^{*}, \psi_{1}, \psi_{2} \mid Y_{1}, Y_{2})
    } {
      \pmeld(\phi, \psi_{1}, \psi_{2} \mid Y_{1}, Y_{2})
    }
    \frac {
      \mathcal{Q}_{1}(\phi)
    } {
      \mathcal{Q}_{1}(\phi^{*})
    } \\[2ex]
  &=
    \frac {
      \pd_{\text{pool}}(\phi^{*})
      \frac {
        \pd_{1}(\phi^{*}, \psi_1, Y_{1})
      } {
        \pd_{1}(\phi^{*})
      }
      \frac {
        \pd_{2}(\phi^{*}, \psi_2, Y_{2})
      } {
        \pd_{2}(\phi^{*})
      }
      \frac {
        \pd_{1}(\phi, \psi_1, Y_{1})
      } {
        \pd_{1}(\phi)
      }
      \tarw_{1}(\phi; \eta_{1})
    } {
      \pd_{\text{pool}}(\phi)
      \frac {
        \pd_{1}(\phi, \psi_1, Y_{1})
      } {
        \pd_{1}(\phi)
      }
      \frac {
        \pd_{2}(\phi, \psi_2, Y_{2})
      } {
        \pd_{2}(\phi)
      }
      \frac {
        \pd_{1}(\phi^{*}, \psi_1, Y_{1})
      } {
        \pd_{1}(\phi^{*})
      }
      \tarw_{1}(\phi^{*}; \eta_{1})
    } \\[2ex]
  &= 
    \frac{
      \pd_{\text{pool}} (\phi^{*})
    } {
      \pd_{\text{pool}} (\phi)
    }
    \frac {
      \pd_{2}(\phi^{*}, \psi_2, Y_{2})
    } {
      \pd_{2}(\phi, \psi_2, Y_{2})
    }
    \frac {
      \pd_{2}(\phi)
    } {
      \pd_{2}(\phi^{*})
    }
    \frac {
      \tarw_{1}(\phi; \eta_{1})
    } {
      \tarw_{1}(\phi^{*}; \eta_{1})
    }
  \label{eqn:stage-two-acceptance}
\end{align}

The presence of the \(\tarw_{1}(\phi; \eta_{1})\) terms in the stage 2
acceptance probability implies that we have some degree of control over
the acceptance rate. The \(\pd_{2}(\phi) \tarw_{1}(\phi; \eta_{1})\)
term also suggests that this idea may be more linked than we suspect to
our weighted estimation methodology for the prior marginal distribution.

For \(\Nm > 2\), we will accrue an additional
\(\tarw_{\modelindex - 1}(\phi; \eta_{\modelindex - 1})\) term for each
additional model, which may be problematic.

\subsection{\texorpdfstring{Choosing
\(\tarw_{1}(\phi; \eta_{1})\)}{Choosing \textbackslash{}tarw\_\{1\}(\textbackslash{}phi; \textbackslash{}eta\_\{1\})}}\label{choosing-tarw_1phi-eta_1}

Say we use the samples of \(\phi\) from
\begin{equation}
  \pd_{\text{meld}, 1}(\phi, \psi_{1} \mid Y_{1}) \propto
    \frac {
      \pd_{1}(\phi, \psi_1, Y_{1})
    } {
      \pd_{1}(\phi)
    }
  \label{eqn:intermediary-meld}
\end{equation} as a proposal for
\(\pd_{\text{meld}}(\phi, \psi_{1}, \psi_{2} \mid Y_{1}, Y_{2})\) and
observe the behaviour in Figure~\ref{fig:no_u_traces}. We can still use
these samples, in combination with samples from
\(\pd_{2}(\phi, \psi_{2} \mid Y_{2})\), to choose an appropriate
\(\tarw_{1}(\phi; \eta_{1})\).

Consider a sample of size \(\Nx\) from the intermediary melded target
\(\pd_{\text{meld}, 1}(\phi, \psi_{1} \mid Y_{1})\) denoted
\(\{\tilde{\phi}_{\sampleindex, 1}\}_{\sampleindex = 1}^{\Nx}\). Using
this sample we compute the sample mean of the first stage target denoted
\(\tilde{\mu}_{1} = \frac{1}{\Nx}\sum\limits_{\sampleindex = 1}^{\Nx}\tilde{\phi}_{\sampleindex, 1}\),
as well as the sample variance
\(\tilde{\sigma}^{2}_{1} = \frac{1}{\Nx - 1}\sum\limits_{\sampleindex = 1}^{\Nx} (\tilde{\phi}_{\sampleindex, 1} - \tilde{\mu}_{1})^{2}\).
Analogous quantities for the second subposterior can then be defined
using the sample of size \(\Nx\) from the second subposterior\footnote{This
  should probably be the stage two target, not the subposterior. Again,
  ignoring the impact of the prior?}
\(\pd_{2}(\phi, \psi_{2} \mid Y_{2})\) as
\(\{\phi_{\sampleindex, 2}\}_{\sampleindex = 1}^{\Nx}\). The sample mean
and variance of the subposterior, \(\hat{\mu}_{2}\) and
\(\hat{\sigma}^{2}_{2}\), are computed in the same manner.

These sample statistics can then be used to define an appropriate
augmentation function \(\tarw_{1}(\phi; \eta_{1})\). We now have to
assume that the intermediary target and subposterior distributions can
be appropriately summarised by Gaussian distributions. Denoting the
Gaussian density function with mean \(\mu\) and variance \(\sigma^2\) as
\(f(\phi; \mu, \sigma^2)\), we define the augmentation function as a
ratio of Gaussian densities
\begin{equation}
  \tarw_{1}(\phi; \eta_{1}) = 
  \frac {
    f(\phi; \mu_{\text{nu}}, \sigma^2_{\text{nu}})
  } {
    f(\phi; \mu_{\text{de}}, \sigma^2_{\text{de}})
  },
  \label{eqn:augmentation-definition}
\end{equation} whose
parameters we will define momentarily. The intuition is that the
denominator density should approximate the intermediary melded target
\(\pd_{\text{meld}, 1}(\phi, \psi_{1} \mid Y_{1})\), whilst the
numerator should approximate
\(\pd_{\text{meld}, 2}(\phi, \psi_{1}, \psi_{2} \mid Y_{1}, Y_{2})\),
which is a combination of both the first stage target and the second
subposterior.

Using this intuition we define the parameters of the denominator density
function to be \(\mu_{\text{de}} = \tilde{\mu}_{1}\) and
\(\sigma^{2}_{\text{de}} = \tilde{\sigma}^{2}_{1}\). We can use the well
known results about products of Gaussian density functions, summarised
in Bromiley (2003), to compute the numerator parameters
\begin{equation}
  \mu_{\text{nu}} = \frac{
    \tilde{\sigma}^{2}_{\modelindex - 1} \hat{\mu}_{\modelindex} + \hat{\sigma}^{2}_{\modelindex}\tilde{\mu}_{\modelindex - 1}
  } {
    \tilde{\sigma}^{2}_{\modelindex - 1} + \hat{\sigma}^{2}_{\modelindex}
  },
  \qquad
  \sigma^{2}_{\text{nu}} = \frac{
    \tilde{\sigma}^{2}_{\modelindex - 1} \hat{\sigma}^{2}_{\modelindex}
  } {
    \tilde{\sigma}^{2}_{\modelindex - 1} + \hat{\sigma}^{2}_{\modelindex}
  },
  \label{eqn:numerator-results}
\end{equation} and for compactness
we define
\(\eta_{1} = \{\mu_{\text{nu}}, \sigma^{2}_{\text{nu}}, \mu_{\text{de}}, \sigma^{2}_{\text{de}}\}\).

In cases where all densities of interest are Gaussian,
\(f(\phi; \mu_{\text{de}}, \sigma^2_{\text{de}})\) precisely summarises
the typical stage one melding target
\(\pd_{1}(\phi, \psi_{1}, Y_{1}) \mathop{/} \pd_{1}(\phi)\). This
results in an augmented target proportional to
\(f(\phi; \mu_{\text{nu}}, \sigma^2_{\text{nu}})\), which we have
defined as a composition of the two distributions of interest.

If \(\pd_{1}(\phi, \psi_{1}, Y_{1}) \mathop{/} \pd_{1}(\phi)\) has
heavier tails than a Gaussian distribution, we may have concerns about
the integrability of the target in
Equation~\eqref{eqn:stage-one-target}. However, by construction
\(\sigma_{\text{nu}}^{2} < \sigma_{\text{de}}^{2}\), which ensures that
\(\tarw_{1}(\phi; \eta_{1})\) is integrable. Given the original target
\(\pd_{1}(\phi, \psi_{1}, Y_{1}) \mathop{/} \pd_{1}(\phi)\) is
integrable\footnote{Not obviously true without pooled prior term.}, and
the augmented target is the product of two integrable densities, the
augmented target is also integrable. This gives the practitioner some
room to tweak \(\tarw_{1}(\phi; \eta_{1})\); as long as
\(\sigma_{\text{nu}}^{2} < \sigma_{\text{de}}^{2}\) is satisfied, the
target is integrable. Thus \(\sigma_{\text{nu}}^{2}\) can be
artificially increased, to a known limit, to partially ameliorate the
impact of the approximation error.

\subsection{\texorpdfstring{\(\Nm > 2\)}{\textbackslash{}Nm \textgreater{} 2}}\label{nm-2}

Applying this methodology in the \(\Nm > 2\) case is awkward.
Disagreement between stage \(\modelindex - 1\) and \(\modelindex\)
requires sampling an augmented version of \(\modelindex - 1\). When
using the multi-stage MH-within-Gibbs setting, this requires resampling
all previous \(\modelindex - 1\) stages. Each of these stages will
require augmenting, in a stage specific way, to ensure that the
interstage conflict is not simply `moved' to earlier stages.

Notationally, the interstage conflict would exist between the
intermediary melded target
\(\pd_{\text{meld}, \modelindex - 1}(\phi, \psi_{1}, \ldots, \psi_{\modelindex - 1}\mid Y_{1}, \ldots, Y_{\modelindex - 1})\)
and the \(m^{\text{th}}\) subposterior
\(\pd_{\modelindex}(\phi, \psi_{\modelindex} \mid Y_{\modelindex})\).
The augmentation function for the \((\modelindex - 1)^{\text{th}}\)
stage target is
\(\tarw_{\modelindex - 1}(\phi; \eta_{\modelindex - 1})\).

\subsection{Similarities to Vehtari et al.
(2014)}\label{similarities-to-vehtarietal14}

Expectation propagation (EP) is interested in approximating a target
density \(f(\phi)\), typically a posterior distribution, with a density
\(g(\phi; \eta)\) from a known parametric family, with parameter
\(\eta\). Both densities are assumed to factorise multiplicatively
\begin{equation}
  f(\phi) \propto 
    \prod_{\modelindex = 0}^{\Nm} f_{\modelindex}(\phi)
  \qquad
  g(\phi; \eta) \propto
    \prod_{\modelindex = 0}^{\Nm} g_{\modelindex}(\phi; \eta_{\modelindex}).
  \label{eqn:factorisations-ep}
\end{equation} It
is assumed that each component of the factorisation of \(f(\phi)\) has
an equivalent component in \(g(\phi; \eta)\). Each component of the
factorisations is referred to as a \emph{site}, and the overall
approximation \(g(\phi; \eta)\) is the \emph{global} approximation.
Vehtari et al. (2014) then define the \emph{cavity distribution}
\(g_{-\modelindex}(\phi; \eta)\) and the \emph{tilted distribution}
\(g_{\setminus\modelindex}(\phi; \eta)\)
\begin{equation}
  g_{-\modelindex}(\phi; \eta) = 
    \frac {
      g(\phi; \eta)
    } {
      g_{\modelindex}(\phi; \eta_{\modelindex})
    }
  \qquad
  g_{\setminus\modelindex}(\phi) = 
    f_{\modelindex}(\phi) g_{-\modelindex}(\phi).
  \label{eqn:ep-distribution-defs}
\end{equation}

Assume we have some initial value for \(\eta\) in \(g(\phi; \eta)\),
which is an initial approximation to the target density. For each site
\(\modelindex\), EP proceeds to

\begin{enumerate}
\def\labelenumi{\arabic{enumi}.}
\tightlist
\item
  Compute the cavity distribution \(g_{-\modelindex}(\phi; \eta)\),
  which is an algebraic calculation for each site as \(g\) is a member
  of an exponential family.
\item
  Improve the site approximation
  \(g_{\modelindex}(\phi; \eta_{\modelindex})\) by optimising
  \(\eta_{\modelindex}\) with respect to a divergence metric, typically
  the Kullback-Leibler (KL) divergence between
  \(f_{\modelindex}(\phi) g_{-\modelindex}(\phi; \eta)\) and
  \(g_{\modelindex}(\phi; \eta_{\modelindex}) g_{-\modelindex}(\phi; \eta)\).
\end{enumerate}

These two steps can be performed in batch-parallel, where each
\(\eta_{\modelindex}\) is optimised in parallel then communicated back
to a central node. The central node computes \(\eta\), then broadcasts
the result back to each parallel node for more optimisation steps, which
is repeated until some convergence criteria is reached. Vehtari et al.
(2014) state that optimising \(\eta_{\modelindex}\) is the critical
operation in EP, and that minimising the KL divergence corresponds to
matching the moments of
\(f_{\modelindex}(\phi) g_{-\modelindex}(\phi; \eta)\) and
\(g_{\modelindex}(\phi; \eta_{\modelindex}) g_{-\modelindex}(\phi; \eta)\).

\subsubsection{Analogies in Markov
melding}\label{analogies-in-markov-melding}

Consider the melded posterior
\begin{align}
  \pd_{\text{meld}}(\phi, \psi_{1}, \ldots, \psi_{\Nm} \mid Y_{1}, \ldots,  Y_{\Nm}) &\propto
    \ppoolphi
    \prod_{\modelindex = 1}^{\Nm}
    \pd_{\modelindex}(\psi_{\modelindex} \mid Y_{\modelindex}, \phi) \\
    &\propto \ppoolphi
    \prod_{\modelindex = 1}^{\Nm}
    \frac {
      \pd_{\modelindex}(\phi, \psi_{\modelindex}, Y_{\modelindex})
    } {
      \pd_{\modelindex}(\phi)
    },
  \label{eqn:melding-factorisation}
\end{align}
noting that the product terms now start at \(\modelindex = 1\). Our
definition of \(\tarw_{\modelindex - 1}(\phi; \eta_{\modelindex - 1})\)
is analogous to the cavity distribution \(g_{-\modelindex}(\phi; \eta)\)
of Vehtari et al. (2014). The denominator density
\(f(\phi; \mu_{\text{de}}, \sigma^{2}_{\text{de}})\) in
Equation~\eqref{eqn:augmentation-definition} is an approximation to
\(\pd_{\modelindex}(\psi_{\modelindex} \mid Y_{\modelindex}, \phi)\) in
precisely the way EP suggests approximating\footnote{This notation
  clashes, I should fix that.} \(f_{\modelindex}(\phi)\) with
\(g_{\modelindex}(\phi; \eta_{\modelindex})\). Likewise, the numerator
density in Equation~\eqref{eqn:augmentation-definition} is acting as the
equivalent \emph{global} approximation \(g(\phi; \eta)\) in EP. There is
a slight difference due to the pooled prior, which we do not\footnote{Suggesting
  I'm not doing something right - would be worth factoring in?. I guess
  I've just been ignoring the impact of the pooled prior.} factor into
the computation of
\(\tarw_{\modelindex - 1}(\phi; \eta_{\modelindex - 1})\).

This suggests adopting the EP strategy to address conflict when melding,
which may be particularly useful in the \(\Nm > 2\) case.

\begin{enumerate}
\def\labelenumi{\arabic{enumi}.}
\tightlist
\item
  Sample \(\ppoolphi\) and the marginalised subposteriors
  \(\pd_{1}(\phi, \psi_{1}, Y_{1}) \mathop{/} \pd_{1}(\phi), \,\, \ldots\,\, , \pd_{\Nm}(\phi, \psi_{\Nm}, Y_{\Nm}) \mathop{/} \pd_{\Nm}(\phi)\)
  in parallel (on separate nodes).
\item
  Approximate each distribution with
  \(g_{\modelindex}(\phi; \eta_{\modelindex})\).
\item
  Send the results back to a central node, and compute the global
  approximation \(g(\phi; \eta)\).
\item
  Broadcast the global approximation back to each node.
\item
  Each node then computes their specific cavity distribution, and
  samples from the resulting tilted distribution.
\item
  Iterate step 3 -- 5 until there is convergence in the global
  approximation.
\end{enumerate}

\subsection{Potential issues}\label{potential-issues}

\begin{itemize}
\tightlist
\item
  Feasible in 1/2/(3?) dimensions, and melding cases where \(\Nm\) is
  \(<10\)?.

  \begin{itemize}
  \tightlist
  \item
    Trying to think of cases where it is easyish to figure out where the
    overall posterior will be by visual inspection.
  \item
    More general idea for finding ``midpoint'' between samples from two
    distributions.

    \begin{itemize}
    \tightlist
    \item
      Seems a lot like the idea of Barycentres in: (Srivastava et al.,
      2015, 2018).
    \item
      Infact, I think this is identical in the 1-D case.
    \item
      In the 2-or-more-D case, we have to start thinking about
      covariance as well.
    \end{itemize}
  \end{itemize}
\item
  Can use ESS as a diagnostic for particle degeneracy?
\item
  In the case of no overlap, we are sensitive to the tails of the
  approximation (in the EP setting).
\end{itemize}

\section{Example: two Gaussians}\label{example-two-gaussians}

Starting from the same Gaussian-Gaussian model used to demonstrate the
importance of accurately estimating the prior marginal distribution, we
adjust some of the hyperparmeters to manufacture disjoint subposteriors.
There are \(\Nm = 2\) models, with common parameter \(\phi\),
\begin{align*}
  Y_{1} \sim \text{N}(\phi, \varepsilon_{1}^{2}) &\quad
  \phi \sim \text{N}(\mu_{1}, \sigma_{1}^2) \quad \modelindex = 1\\
  Y_{2} \sim \text{N}(\phi, \varepsilon_{2}^{2}) &\quad
  \phi \sim \text{N}(\mu_{2}, \sigma_{2}^2)\quad \modelindex = 2.
\end{align*} We simulate 5
observations from model 1, i.e.~\(Y_{1}\) is a vector of length 5, and 5
observations from model 2. The observations variances
\(\varepsilon_{1}^{2}\) and \(\varepsilon_{2}^{2}\) are fixed to 1 and 1
respectively. The prior means \(\mu_{1}\) and \(\mu_{2}\) are -2 and 2,
whilst the prior variances are 1 and 1. This configuration results in
the subposterior distributions in Figure~\ref{fig:subposteriors}.

\begin{figure}

{\centering \includegraphics[width=1\linewidth]{plots/norm-norm-ex/subposteriors} 

}

\caption{Disjoint subposteriors for the two Gaussian example.}\label{fig:subposteriors}
\end{figure}

The pooled prior is formed by linear pooling. For the mulit-stage
samplers, the stage one target is defined in
Equation~\eqref{eqn:stage-one-target}.

\subsection{\texorpdfstring{Stage one target and
\(\text{u}_{1}(\phi)\).}{Stage one target and \textbackslash{}text\{u\}\_\{1\}(\textbackslash{}phi).}}\label{stage-one-target-and-textu_1phi.}

Here we choose \(\text{u}_{1}(\phi)\) via the method described in
Section~\ref{choosing-tarw_1phi-eta_1}.
Figure~\ref{fig:u_func_augmented_target} displays \(\text{u}_{1}(\phi)\)
(right panel) and the augmented target
\(\tilde{\pd}_{\text{meld}, 1}(\phi, \psi_{1}, Y_{1})\) (left panel).
The augmented target now lies inbetween the two subposteriors depicted
in Figure~\ref{fig:subposteriors}.

\begin{figure}

{\centering \includegraphics[width=1\linewidth]{plots/norm-norm-ex/u-function-augmented-target} 

}

\caption{The augmented stage one target (left panel) and corresponding $\text{u}_{1}(\phi)$ (right panel).}\label{fig:u_func_augmented_target}
\end{figure}

\subsection{Numerical tests}\label{numerical-tests}

We want to compare the melded posterior distribution sampled using the
following methods:

\begin{itemize}
\tightlist
\item
  Multi-stage sampler, no augmentation function (standard melding).
\item
  Multi-stage sampler, \emph{with} augmentation function.
\item
  Directly sampling the melded posterior, as a point of reference.
\end{itemize}

We can analytically compute the melded posterior, however evaluating is
numerically challenging. I have yet to find the calculations that
introduce the numerical inaccuracy I observe.

\paragraph{\texorpdfstring{Sequential MH, with no \(\text{u}_{1}(\phi)\)
(standard
melding)}{Sequential MH, with no \textbackslash{}text\{u\}\_\{1\}(\textbackslash{}phi) (standard melding)}}\label{sequential-mh-with-no-textu_1phi-standard-melding}

Here we apply the multi-stage sampler with no augmentation function, or
equivalently set \(\tarw_{1}(\phi; \eta_{1}) = 1\). The traceplots in
Figure~\ref{fig:no_u_traces} demonstrate the issue with this approach,
specifically that the support of the first stage does not intersect with
the support of the melded posterior.

\begin{figure}

{\centering \includegraphics[width=1\linewidth]{plots/norm-norm-ex/no-u/stage-traces} 

}

\caption{Traceplots for the first and second stage target distributions, with no subposterior augmentation.}\label{fig:no_u_traces}
\end{figure}

\paragraph{\texorpdfstring{Sequential MH, \emph{with}
\(\text{u}_{1}(\phi)\) (idea developed
here)}{Sequential MH, with \textbackslash{}text\{u\}\_\{1\}(\textbackslash{}phi) (idea developed here)}}\label{sequential-mh-with-textu_1phi-idea-developed-here}

Now we sample the augmented target in the first stage, for use in the
second and final stage. Figure~\ref{fig:with_u_traces_2} shows a clear
improvement in performance over Figure~\ref{fig:no_u_traces}.

\begin{figure}

{\centering \includegraphics[width=1\linewidth]{plots/norm-norm-ex/with-u-2/stage-traces} 

}

\caption{Traceplots for the fist and second stage targets distributions, \textit{with} augmented stage one target.}\label{fig:with_u_traces_2}
\end{figure}

\paragraph{\texorpdfstring{Joint MH - sampling
\(\pmeld(\phi \mid Y_{1}, Y_{2})\)
directly}{Joint MH - sampling \textbackslash{}pmeld(\textbackslash{}phi \textbackslash{}mid Y\_\{1\}, Y\_\{2\}) directly}}\label{joint-mh---sampling-pmeldphi-mid-y_1-y_2-directly}

In this example we can specify the melded posterior in Stan directly.
Figure~\ref{fig:joint_trace} is the traceplot of the sampler that
directly targets the melded posterior.

\begin{figure}

{\centering \includegraphics[width=1\linewidth]{plots/norm-norm-ex/joint/trace} 

}

\caption{Traceplot of the directly sampled melded posterior.}\label{fig:joint_trace}
\end{figure}

We compare the sample quantiles of the melded joint posterior and the
stage two melded posterior obtained using the augmented stage one target
in Figure~\ref{fig:joint_augmented_compare}, and we see good agreement
between the two samples.

\begin{figure}

{\centering \includegraphics[width=1\linewidth]{plots/norm-norm-ex/joint-augmented-compare} 

}

\caption{Quantile-quantile plot of sample quantiles obtained by directly sampling the melded joint posterior (y-axis) and the stage two sample quantiles obtained using the augmented stage one target (x-axis).}\label{fig:joint_augmented_compare}
\end{figure}

\section{Computational costs}\label{computational-costs}

\subsection{Multi-stage Metropolis-Hastings within
Gibbs}\label{multi-stage-metropolis-hastings-within-gibbs}

\subsubsection*{Two submodel case}\label{two-submodel-case}
\addcontentsline{toc}{subsubsection}{Two submodel case}

We would like to generate \(\Nx\) samples from the overall melded
posterior, and hence opt to generate \(\Nx\) samples from the stage one
target. Each stage one sample has a computational cost of order
\(\mathcal{O}(C_{1})\) which involves evaluating the unnormalised
logposterior (\(\mathcal{O}(C_{1, \text{lp}})\)), and generating and
evaluating a proposal for the link parameter
(\(\mathcal{O}(C_{1, \mathcal{Q}})\)). Note that
\(\mathcal{O}(C_{1}) = \mathcal{O}(C_{1, \text{lp}} + C_{1, \mathcal{Q}})\).
The equivalent set of costs exist for the second stage target.

First we have to identify that we have a problem:

\begin{itemize}
\tightlist
\item
  The cost of sampling the original stage one target:
  \(\mathcal{O}(\Nx C_{1})\).
\item
  The cost of evaluating said samples in stage two:
  \(\mathcal{O}(\Nx C_{2, \text{lp}})\).
\end{itemize}

Now we can solve it by:

\begin{itemize}
\tightlist
\item
  Sampling the second stage subposterior: \(\mathcal{O}(\Nx C_{2})\).

  \begin{itemize}
  \tightlist
  \item
    This does not include the pooled prior.
  \end{itemize}
\item
  Sampling the augmented stage one target: \(\mathcal{O}(\Nx C_{1}')\).

  \begin{itemize}
  \tightlist
  \item
    The prime indicates that the augmented target has strictly greater
    computational cost to evaluate.
  \end{itemize}
\item
  Evaluating the augmented samples in stage two:
  \(\mathcal{O}(\Nx C_{2, \text{lp}})\).
\end{itemize}

\subsubsection*{More than two submodels}\label{more-than-two-submodels}
\addcontentsline{toc}{subsubsection}{More than two submodels}

Fri 26 Jul 13:16:49 2019: \emph{I don't think this is correct, consider
three subposteriors in a triangle.}

If we have \(\Nm > 2\) models, then the costs grow in the following way.

Identifying the problem:

\begin{itemize}
\tightlist
\item
  We finish sampling the \(\modelindex-1\)th stage target:
  \(\mathcal{O}(\Nx (C_{1} + C_{2, \text{lp}} + \ldots + C_{\modelindex - 1, \text{lp}}))\).
\item
  Then, we evaluate these samples under the \(\modelindex\)th stage:
  \(\mathcal{O}(\Nx C_{\modelindex, \text{lp}})\) and realise we have
  conflict.
\end{itemize}

Addressing the issue:

\begin{itemize}
\tightlist
\item
  Sampling all the subposteriors:
  \(\mathcal{O}(\Nx (C_{1} + \ldots + C_{\Nm}))\)
\item
  Sampling the augmented stage one target \(\mathcal{O}(\Nx C_{1}')\).
\item
  Evaluating them under the augmented intermediary targets:
  \(\mathcal{O}(\Nx (C_{2, \text{lp}}' + \ldots + C_{\Nm, \text{lp}}))\).

  \begin{itemize}
  \tightlist
  \item
    Unless we augmented all the intermediary targets, any one stage
    could render the sample degenerate.
  \item
    We do \textbf{not} augment the final target.
  \end{itemize}
\end{itemize}

\subsection{Sequential Monte Carlo
(SMC)}\label{sequential-monte-carlo-smc}

Adopting an SMC approach alleviates the need for initial runs to both
diagnose conflict, and calculate an appropriate augmentation function
\(\tarw_{1}(\phi; \eta_{1})\). Consider the \(\Nm = 2\) models case
initially. If introducing each submodel requires \(T\) tempering steps,
each with a single refreshment step, and we wish to generate \(\Nx\)
samples, the computational costs include:

\begin{itemize}
\tightlist
\item
  Sampling the stage one target: \(\mathcal{O}(\Nx C_{1})\).
\item
  Evaluating the \(T\) tempered likelihoods, and then refreshing the
  particles: \(\mathcal{O}(\Nx T (C_{1} + C_{2}))\). 
\item
  The overall cost is then \(\mathcal{O}(\Nx T (2 C_{1} + C_{2}))\).
\end{itemize}

We can extrapolate this to \(\Nm > 2\) models. Assume introducing each
additional submodel also requires \(T\) tempering steps. The final
computational cost is then
\(\mathcal{O}(\Nx T (\Nm C_{1} + (\Nm - 1) C_{2} + \ldots + C_{\Nm}))\).
Perhaps there are ways to reduce the number of times we need to evaluate
the previous models when introducing the \(\modelindex\)th model. A
careful reading of (Lindsten et al., 2017) seems prudent, as the naive
SMC algorithm I am envisaging here involves evaluating the joint model
(i.e.~all the data).

\section{Example - real world? Where can we get disjoint conflict
from.}\label{example---real-world-where-can-we-get-disjoint-conflict-from.}

\begin{itemize}
\tightlist
\item
  Investigate the Carlin example Rob sent
\end{itemize}

\newpage

\section*{Bibliography}\label{bibliography}
\addcontentsline{toc}{section}{Bibliography}

\hypertarget{refs}{}
\hypertarget{ref-bromiley:03}{}
Bromiley, P., 2003. Products and Convolutions of Gaussian Probability
Density Functions\emph{. Tina-Vision Memo} 3, 1.

\hypertarget{ref-lindsten:etal:17}{}
Lindsten, F., Johansen, A.M., Naesseth, C.A., Kirkpatrick, B., Schön,
T.B., Aston, J.A.D., Bouchard-Côté, A., 2017. Divide-and-Conquer with
Sequential Monte Carlo\emph{. Journal of Computational and Graphical
Statistics} 26, 445--458.
\url{https://doi.org/10.1080/10618600.2016.1237363}

\hypertarget{ref-srivastava:etal:15}{}
Srivastava, S., Cevher, V., Dinh, Q., Dunson, D., 2015. WASP: Scalable
Bayes via barycenters of subset posteriors\emph{, in: Lebanon, G.,
Vishwanathan, S.V.N. (Eds.), Proceedings of the Eighteenth International
Conference on Artificial Intelligence and Statistics, Proceedings of
Machine Learning Research}. PMLR, San Diego, California, USA, pp.
912--920.

\hypertarget{ref-srivastava:li:dunson:18}{}
Srivastava, S., Li, C., Dunson, D.B., 2018. Scalable Bayes via
Barycenter in Wasserstein Space\emph{. J. Mach. Learn. Res.} 19,
312--346.

\hypertarget{ref-vehtari:etal:14}{}
Vehtari, A., Gelman, A., Sivula, T., Jylänki, P., Tran, D., Sahai, S.,
Blomstedt, P., Cunningham, J.P., Schiminovich, D., Robert, C., 2014.
Expectation Propagation as a Way of Life: A Framework for Bayesian
Inference on Partitioned Data\emph{. arXiv e-prints} arXiv:1412.4869.

\newpage

\renewcommand{\thesection}{\Alph{section}}

\setcounter{section}{0}

\section{Appendix}\label{appendix}


\end{document}
