\documentclass[10pt,a4paper,]{article}
\usepackage[]{tgpagella}
\usepackage{amssymb,amsmath}
\usepackage{ifxetex,ifluatex}
\usepackage{fixltx2e} % provides \textsubscript
\ifnum 0\ifxetex 1\fi\ifluatex 1\fi=0 % if pdftex
  \usepackage[T1]{fontenc}
  \usepackage[utf8]{inputenc}
\else % if luatex or xelatex
  \ifxetex
    \usepackage{mathspec}
  \else
    \usepackage{fontspec}
  \fi
  \defaultfontfeatures{Ligatures=TeX,Scale=MatchLowercase}
\fi
% use upquote if available, for straight quotes in verbatim environments
\IfFileExists{upquote.sty}{\usepackage{upquote}}{}
% use microtype if available
\IfFileExists{microtype.sty}{%
\usepackage{microtype}
\UseMicrotypeSet[protrusion]{basicmath} % disable protrusion for tt fonts
}{}
\usepackage[margin=2.25cm]{geometry}
\usepackage{hyperref}
\hypersetup{unicode=true,
            pdftitle={Addressing Subposterior Conflict With Weighted Proposals},
            pdfauthor={Andrew Manderson},
            pdfborder={0 0 0},
            breaklinks=true}
\urlstyle{same}  % don't use monospace font for urls
\usepackage{graphicx,grffile}
\makeatletter
\def\maxwidth{\ifdim\Gin@nat@width>\linewidth\linewidth\else\Gin@nat@width\fi}
\def\maxheight{\ifdim\Gin@nat@height>\textheight\textheight\else\Gin@nat@height\fi}
\makeatother
% Scale images if necessary, so that they will not overflow the page
% margins by default, and it is still possible to overwrite the defaults
% using explicit options in \includegraphics[width, height, ...]{}
\setkeys{Gin}{width=\maxwidth,height=\maxheight,keepaspectratio}
\IfFileExists{parskip.sty}{%
\usepackage{parskip}
}{% else
\setlength{\parindent}{0pt}
\setlength{\parskip}{6pt plus 2pt minus 1pt}
}
\setlength{\emergencystretch}{3em}  % prevent overfull lines
\providecommand{\tightlist}{%
  \setlength{\itemsep}{0pt}\setlength{\parskip}{0pt}}
\setcounter{secnumdepth}{5}
% Redefines (sub)paragraphs to behave more like sections
\ifx\paragraph\undefined\else
\let\oldparagraph\paragraph
\renewcommand{\paragraph}[1]{\oldparagraph{#1}\mbox{}}
\fi
\ifx\subparagraph\undefined\else
\let\oldsubparagraph\subparagraph
\renewcommand{\subparagraph}[1]{\oldsubparagraph{#1}\mbox{}}
\fi

%%% Use protect on footnotes to avoid problems with footnotes in titles
\let\rmarkdownfootnote\footnote%
\def\footnote{\protect\rmarkdownfootnote}

%%% Change title format to be more compact
\usepackage{titling}

% Create subtitle command for use in maketitle
\providecommand{\subtitle}[1]{
  \posttitle{
    \begin{center}\large#1\end{center}
    }
}

\setlength{\droptitle}{-2em}

  \title{Addressing Subposterior Conflict With Weighted Proposals}
    \pretitle{\vspace{\droptitle}\centering\huge}
  \posttitle{\par}
    \author{Andrew Manderson}
    \preauthor{\centering\large\emph}
  \postauthor{\par}
      \predate{\centering\large\emph}
  \postdate{\par}
    \date{23 July, 2019}

\usepackage{amsmath}
% I always seem to need tikz for something
\usepackage{tikz}
\usetikzlibrary{positioning, shapes, intersections, through, backgrounds, fit, decorations.pathmorphing}

\usepackage{setspace}
\onehalfspacing

\usepackage{lineno}
% \linenumbers

% required for landscape pages. beware, they back the build very slow.
\usepackage{pdflscape}

% table - `gt' package uses these, often unimportant
% \usepackage{longtable}
% \usepackage{booktabs}
% \usepackage{caption}

\setcounter{secnumdepth}{3}

% pd stands for: probability distribution and is useful to distringuish
% marignals for probabilities specifically p(p_{1}) and the like.
\newcommand{\pd}{\text{p}}
\newcommand{\q}{\text{q}}
\newcommand{\w}{\text{w}}
\newcommand{\pdr}{\text{r}}
\newcommand{\pdrh}{\hat{\text{r}}}

% melding
\newcommand{\ppoolphi}{\pd_{\text{pool}}(\phi)}
\newcommand{\pmeld}{\pd_{\text{meld}}}

% the q(x)w(x), "weighted target" density 
% for the moment I'm going to call it s(x), as that is the next letter of the 
% alphabet. Can change it later
\newcommand{\s}{\text{s}}
% direct density estimate - replaces lambda.
\newcommand{\ddest}{\text{s}}
% target weighting function
\newcommand{\tarw}{\text{u}}

% constants - usually sizes of things
\newcommand{\Nx}{N}
\newcommand{\Nnu}{\text{N}_{\text{nu}}}
\newcommand{\Nde}{\text{N}_{\text{de}}}
\newcommand{\Nmc}{\text{N}_{\text{mc}}}
\newcommand{\Nw}{W}
\newcommand{\Nm}{M}

% locales - could switch to x and x'
\newcommand{\xnu}{x_{\text{nu}}}
\newcommand{\xde}{x_{\text{de}}}

% sugiyama stuff
\newcommand{\pdnu}{\pd_{\text{nu}}}
\newcommand{\pdde}{\pd_{\text{de}}}

% indices 
\newcommand{\wfindex}{w}
\newcommand{\sampleindex}{n}
\newcommand{\modelindex}{m}

%% operators that need additional spacing
\newcommand{\wpropto}{\,\,\propto\,\,}

\begin{document}
\maketitle

The need for accurate self-density ratio estimates arose from a conflict
between \((m - 1)^{\text{th}}\) stage posterior and the unknown
\(m^{\text{th}}\) model's prior marginal distribution. This was purely a
disagreement of scale, the distributions in question did not have
disjoint supports. This document considers subposteriors with disjoint
supports, and demonstrates some potential solutions.

\section{Disjoint subposteriors}\label{disjoint-subposteriors}

Consider the two subposteriors \(\pd_{1}(\phi \mid Y_{1})\),
\(\pd_{2}(\phi \mid Y_{2})\), and melded posterior
\(\pmeld(\phi \mid Y_{1}, Y_{2})\) shown in
Figure~\ref{fig:subpost_disagreement}. Using the samples of either of
the subposteriors as a proposal for the overall posterior results in a
degenerate sample of the overall posterior, due to an insufficient
number of samples in the support of \(\pmeld(\phi \mid, Y_{1}, Y_{2})\).
If instead we sample a version of one of the subposteriors which is
\emph{augmented} by a function \(\tarw_{1}(\phi; \eta_{1})\),
i.e.~\(\pd_{1}(\phi \mid Y_{1})\tarw_{1}(\phi; \eta_{1})\), we can
control the stage two acceptance probability through
\(\tarw_{1}(\phi; \eta_{1})\). The green line in
Figure~\ref{fig:subpost_disagreement} illustrates how we can suitably
scale and shift the stage one target.

\begin{figure}

{\centering \includegraphics[width=1\linewidth]{plots/intro/subpost-disagreement} 

}

\caption{Hypothetical conflicting subposteriors (blue), melded posterior (red), and the augmented stage one one target (green, dashed).}\label{fig:subpost_disagreement}
\end{figure}

The general problem concerns instances where the support of
\(\pd_{\text{meld}, \modelindex - 1}(\phi, \psi_{1}, \ldots, \psi_{\modelindex - 1} \mid Y_{1}, \ldots, Y_{\modelindex - 1})\)
does not intersect with the support of
\(\pd_{\text{meld}, \modelindex}(\phi, \psi_{1}, \ldots, \psi_{\modelindex} \mid Y_{1}, \ldots, Y_{\modelindex})\),
hence also does not overlap with
\(\pd_{\modelindex}(\phi,~\psi_{\modelindex}~\mid~Y_{\modelindex})\). We
must choose the functional form of
\(\tarw_{\modelindex - 1}(\phi; \eta_{\modelindex - 1})\), as well as
the parameter \(\eta_{\modelindex - 1}\), such that the supports of
\(\pd_{\text{meld}, \modelindex - 1}(\phi, \psi_{1}, \ldots, \psi_{\modelindex - 1} \mid Y_{1}, \ldots, Y_{\modelindex - 1})\tarw_{\modelindex - 1}(\phi; \eta_{\modelindex - 1})\)
and
\(\pd_{\text{meld}, \modelindex}(\phi, \psi_{1}, \ldots, \psi_{\modelindex} \mid Y_{1}, \ldots, Y_{\modelindex})\)
intersect.

\section{Methods - Maths}\label{methods---maths}

Consider the overall melded target posterior for \(\Nm = 2\) models
\begin{equation}
  \pmeld (\phi, \psi_{1}, \psi_{2} \mid Y_{1}, Y_{2}) \wpropto
  \ppoolphi
  \frac{
    \pd_{1}(\phi, \psi_1, Y_{1})
  } {
    \pd_{1}(\phi)
  }
  \frac{
    \pd_{2}(\phi, \psi_2, Y_{2})
  } {
    \pd_{2}(\phi)
  }.
  \label{eqn:melded-target-posterior}
\end{equation} The
\emph{augmented stage one target} is then
\begin{equation}
  \pd_{\text{meld}, 1}(\phi, \psi_{1}, Y_{1}) \wpropto
  \frac {
    \pd_{1}(\phi, \psi_1, Y_{1})
  } {
    \pd_{1}(\phi)
  }
  \tarw_{1}(\phi; \eta_{1})
  =
  \mathcal{Q}_{1}(\phi).
  \label{eqn:stage-one-target}
\end{equation} Hence, the stage two
acceptance probability is
\begin{align}
 \alpha(\phi^{*}, \phi) 
  &= 
    \frac {
      \pmeld(\phi^{*}, \psi_{1}, \psi_{2} \mid Y_{1}, Y_{2})
    } {
      \pmeld(\phi, \psi_{1}, \psi_{2} \mid Y_{1}, Y_{2})
    }
    \frac {
      \mathcal{Q}_{1}(\phi)
    } {
      \mathcal{Q}_{1}(\phi^{*})
    } \\[2ex]
  &=
    \frac {
      \pd_{\text{pool}}(\phi^{*})
      \frac {
        \pd_{1}(\phi^{*}, \psi_1, Y_{1})
      } {
        \pd_{1}(\phi^{*})
      }
      \frac {
        \pd_{2}(\phi^{*}, \psi_2, Y_{2})
      } {
        \pd_{2}(\phi^{*})
      }
      \frac {
        \pd_{1}(\phi, \psi_1, Y_{1})
      } {
        \pd_{1}(\phi)
      }
      \tarw_{1}(\phi; \eta_{1})
    } {
      \pd_{\text{pool}}(\phi)
      \frac {
        \pd_{1}(\phi, \psi_1, Y_{1})
      } {
        \pd_{1}(\phi)
      }
      \frac {
        \pd_{2}(\phi, \psi_2, Y_{2})
      } {
        \pd_{2}(\phi)
      }
      \frac {
        \pd_{1}(\phi^{*}, \psi_1, Y_{1})
      } {
        \pd_{1}(\phi^{*})
      }
      \tarw_{1}(\phi^{*}; \eta_{1})
    } \\[2ex]
  &= 
    \frac{
      \pd_{\text{pool}} (\phi^{*})
    } {
      \pd_{\text{pool}} (\phi)
    }
    \frac {
      \pd_{2}(\phi^{*}, \psi_2, Y_{2})
    } {
      \pd_{2}(\phi, \psi_2, Y_{2})
    }
    \frac {
      \pd_{2}(\phi)
    } {
      \pd_{2}(\phi^{*})
    }
    \frac {
      \tarw_{1}(\phi; \eta_{1})
    } {
      \tarw_{1}(\phi^{*}; \eta_{1})
    }
  \label{eqn:stage-two-acceptance}
\end{align}

The presence of the \(\tarw_{1}(\phi; \eta_{1})\) terms in the stage 2
acceptance probability implies that we have some degree of control over
the acceptance rate. The \(\pd_{2}(\phi) \tarw_{1}(\phi; \eta_{1})\)
term also suggests that this idea may be more linked than we suspect to
our weighted estimation methodology for the prior marginal distribution.

For \(\Nm > 2\), we will accrue an additional
\(\tarw_{\modelindex - 1}(\phi; \eta_{\modelindex - 1})\) term for each
additional model, which may be problematic.

\subsection{\texorpdfstring{Choosing
\(\tarw_{\modelindex}(\phi; \eta_{\modelindex})\)}{Choosing \textbackslash{}tarw\_\{\textbackslash{}modelindex\}(\textbackslash{}phi; \textbackslash{}eta\_\{\textbackslash{}modelindex\})}}\label{choosing-tarw_modelindexphi-eta_modelindex}

\subsubsection{Using summary statistics, assuming distribution is
Normal}\label{using-summary-statistics-assuming-distribution-is-normal}

Say we use the samples of \(\phi\) from
\(\pd_{\text{meld}, \modelindex - 1}(\phi, \psi_{1}, \ldots, \psi_{\modelindex - 1} \mid Y_{1}, \ldots, Y_{\modelindex - 1})\)
as a proposal for
\(\pd_{\text{meld}, \modelindex}(\phi, \psi_{1}, \ldots, \psi_{\modelindex} \mid Y_{1}, \ldots, Y_{\modelindex})\)
and observe the behaviour in Figure~\ref{fig:no_u_traces}. We can still
use these samples, in combination with samples from
\(\pd_{\modelindex}(\phi, \psi_{\modelindex} \mid Y_{\modelindex})\), to
choose an appropriate
\(\tarw_{\modelindex - 1}(\phi; \eta_{\modelindex - 1})\).

Define:

\begin{itemize}
\item
  \(\{\tilde{\phi}_{\sampleindex, \modelindex - 1}\}_{\sampleindex = 1, \ldots, \Nx}\)
  is a sample of size \(\Nx\) from
  \(\pd_{\text{meld}, \modelindex - 1}(\phi, \psi_{1}, \ldots, \psi_{\modelindex - 1} \mid Y_{1}, \ldots, Y_{\modelindex - 1})\).

  \begin{itemize}
  \tightlist
  \item
    I am using the tilde to differentiate between samples from the
    intermediary targets and samples from individual subposteriors.
  \end{itemize}
\item
  \(\tilde{\mu}_{\modelindex - 1} = \frac{1}{\Nx}\sum\limits_{i = 1}^{\Nx}\tilde{\phi}_{\sampleindex, \modelindex - 1}\),
  the empirical mean of the \((\modelindex - 1)^{\text{th}}\) stage
  melded posterior.
\item
  \(\tilde{\sigma}^{2}_{\modelindex - 1} = \frac{1}{\Nx - 1}\sum\limits_{i = 1}^{\Nx} (\tilde{\phi}_{\sampleindex, \modelindex - 1} - \tilde{\mu}_{\modelindex - 1})^{2}\),
  the empirical variance of the \((\modelindex - 1)^{\text{th}}\) stage
  melded posterior.
\end{itemize}

Likewise,

\begin{itemize}
\tightlist
\item
  \(\{\phi_{\sampleindex, \modelindex}\}_{\sampleindex = 1, \ldots, \Nx}\)
  is a sample of size \(\Nx\) from
  \(\pd_{\modelindex}(\phi, \psi_{\modelindex} \mid Y_{\modelindex})\).
\item
  \(\mu_{\modelindex} = \frac{1}{\Nx}\sum\limits_{i = 1}^{\Nx}\phi_{\sampleindex, \modelindex}\),
  the empirical mean of the \(\modelindex^{\text{th}}\) subposterior.
\item
  \(\sigma^{2}_{\modelindex} = \frac{1}{\Nx - 1}\sum\limits_{i = 1}^{\Nx} (\phi_{\sampleindex, \modelindex} - \mu_{\modelindex})^{2}\),
  the empirical variance of the \(\modelindex^{\text{th}}\)
  subposterior.
\end{itemize}

Next,

\begin{itemize}
\item
  \(k = 1.05\), some safety factor.

  \begin{itemize}
  \tightlist
  \item
    We need some safety factor to ensure the target is integrable.
  \item
    The integrability of the resulting target is a complicated function
    of the posterior and the space of possible
    \(\tarw_{\modelindex - 1}(\phi; \eta_{\modelindex - 1})\) functions.
  \item
    The tail behaviour is going to be important again.

    \begin{itemize}
    \tightlist
    \item
      Seems to be a recurrent theme.
    \end{itemize}
  \end{itemize}
\item
  \(\mu_{\text{de}} = \tilde{\mu}_{\modelindex - 1}\)
\item
  \(\sigma^{2}_{\text{de}} = k \tilde{\sigma}^{2}_{\modelindex - 1}\)
\item
  \(\mu_{\text{nu}} = \frac{\tilde{\sigma}^{2}_{\modelindex - 1} \mu_{\modelindex} + \sigma^{2}_{\modelindex}\tilde{\mu}_{\modelindex - 1}} {\tilde{\sigma}^{2}_{\modelindex - 1} + \sigma^{2}_{\modelindex}}\)
\item
  \(\sigma^{2}_{\text{nu}} = \frac{\tilde{\sigma}^{2}_{\modelindex - 1} \sigma^{2}_{\modelindex}}{\tilde{\sigma}^{2}_{\modelindex - 1} + \sigma^{2}_{\modelindex}}\)
\end{itemize}

Denoting \(f(\phi; \mu, \sigma^2)\) as the normal density function with
mean \(\mu\) and variance \(\sigma^2\), we can now define

\begin{itemize}
\tightlist
\item
  \(\tarw_{\modelindex - 1}(\phi; \eta_{\modelindex - 1}) = \frac{f(\phi; \mu_{\text{nu}}, \sigma^2_{\text{nu}})} {f(\phi; \mu_{\text{de}}, \sigma^2_{\text{de}})}\).
\item
  Implies
  \(\eta_{\modelindex - 1} = \{\mu_{\text{nu}}, \sigma^{2}_{\text{nu}}, \mu_{\text{de}}, \sigma^{2}_{\text{de}}\}\)
\end{itemize}

\subsection{Potential issues}\label{potential-issues}

\begin{itemize}
\tightlist
\item
  Feasible in 1/2/(3?) dimensions, and melding cases where \(\Nm\) is
  \(<10\)?.

  \begin{itemize}
  \tightlist
  \item
    Trying to think of cases where it is easyish to figure out where the
    overall posterior will be by visual inspection.
  \item
    More general idea for finding ``midpoint'' between samples from two
    distributions.

    \begin{itemize}
    \tightlist
    \item
      Seems a lot like the idea of Barycentres in: (Srivastava et al.,
      2015, 2018).
    \item
      Infact, I think this is identical in the 1-D case.
    \item
      In the 2-or-more-D case, we have to start thinking about
      covariance as well.
    \end{itemize}
  \end{itemize}
\item
  Can use ESS as a diagnostic for particle degeneracy?
\end{itemize}

\section{Example - normal normal}\label{example---normal-normal}

Starting from the same normal-normal model used to demonstrate the
importance of accurately estimating the prior marginal distribution, we
adjust some of the hyperparmeters to manufacture disjoint subposteriors.
There are \(\Nm = 2\) models, with common parameter \(\phi\),
\begin{align*}
  Y_{1} \sim \text{N}(\phi, \varepsilon_{1}^{2}) &\quad
  \phi \sim \text{N}(\mu_{1}, \sigma_{1}^2) \quad \modelindex = 1\\
  Y_{2} \sim \text{N}(\phi, \varepsilon_{2}^{2}) &\quad
  \phi \sim \text{N}(\mu_{2}, \sigma_{2}^2)\quad \modelindex = 2.
\end{align*} We simulate 5
observations from model 1, i.e.~\(Y_{1}\) is a vector of length 5, and 5
observations from model 2. The observations variances
\(\varepsilon_{1}^{2}\) and \(\varepsilon_{2}^{2}\) are fixed to 1 and 1
respectively. The prior means \(\mu_{1}\) and \(\mu_{2}\) are -2 and 2,
whilst the prior variances are 1 and 1. This configuration results in
the subposterior distributions in Figure~\ref{fig:subposteriors}.

\begin{figure}

{\centering \includegraphics[width=1\linewidth]{plots/norm-norm-ex/subposteriors} 

}

\caption{Disjoint subposteriors for the normal normal example.}\label{fig:subposteriors}
\end{figure}

The pooled prior is formed by linear pooling. For the mulit-stage
samplers, the stage one target is defined in
Equation~\eqref{eqn:stage-one-target}.

\subsection{\texorpdfstring{Stage one target and
\(\text{u}_{1}(\phi)\).}{Stage one target and \textbackslash{}text\{u\}\_\{1\}(\textbackslash{}phi).}}\label{stage-one-target-and-textu_1phi.}

Here we choose \(\text{u}_{1}(\phi)\) via the method described in
Section~\ref{choosing-tarw_modelindexphi-eta_modelindex}.
Figure~\ref{fig:u_func_augmented_target} displays \(\text{u}_{1}(\phi)\)
(right panel) and the augmented target
\(\tilde{\pd}_{\text{meld}, 1}(\phi, \psi_{1}, Y_{1})\) (left panel).
The augmented target now lies inbetween the two subposteriors depicted
in Figure~\ref{fig:subposteriors}.

\begin{figure}

{\centering \includegraphics[width=1\linewidth]{plots/norm-norm-ex/u-function-augmented-target} 

}

\caption{The augmented stage one target (left panel) and corresponding $\text{u}_{1}(\phi)$ (right panel).}\label{fig:u_func_augmented_target}
\end{figure}

\subsection{Numerical tests}\label{numerical-tests}

We want to compare the melded posterior distribution sampled using the
following methods:

\begin{itemize}
\tightlist
\item
  Multi-stage sampler, no augmentation function (normal melding).
\item
  Multi-stage sampler, \emph{with} augmentation function.
\item
  Directly sampling the melded posterior, as a point of reference.
\end{itemize}

We can analytically compute the melded posterior, however evaluating is
numerically challenging. I have yet to find the calculations that
introduce the numerical inaccuracy I observe.

\paragraph{\texorpdfstring{Sequential MH, with no \(\text{u}_{1}(\phi)\)
(normal
melding)}{Sequential MH, with no \textbackslash{}text\{u\}\_\{1\}(\textbackslash{}phi) (normal melding)}}\label{sequential-mh-with-no-textu_1phi-normal-melding}

Here we apply the multi-stage sampler with no augmentation function, or
equivalently set \(\tarw_{1}(\phi; \eta_{1}) = 1\). The traceplots in
Figure~\ref{fig:no_u_traces} demonstrate the issue with this approach,
specifically that the support of the first stage does not intersect with
the support of the melded posterior.

\begin{figure}

{\centering \includegraphics[width=1\linewidth]{plots/norm-norm-ex/no-u/stage-traces} 

}

\caption{Traceplots for the first and second stage target distributions, with no subposterior augmentation.}\label{fig:no_u_traces}
\end{figure}

\paragraph{\texorpdfstring{Sequential MH, \emph{with}
\(\text{u}_{1}(\phi)\) (idea developed
here)}{Sequential MH, with \textbackslash{}text\{u\}\_\{1\}(\textbackslash{}phi) (idea developed here)}}\label{sequential-mh-with-textu_1phi-idea-developed-here}

Now we sample the augmented target in the first stage, for use in the
second and final stage. Figure~\ref{fig:with_u_traces_2} shows a clear
improvement in performance over Figure~\ref{fig:no_u_traces}.

\begin{figure}

{\centering \includegraphics[width=1\linewidth]{plots/norm-norm-ex/with-u-2/stage-traces} 

}

\caption{Traceplots for the fist and second stage targets distributions, \textit{with} augmented stage one target.}\label{fig:with_u_traces_2}
\end{figure}

\paragraph{\texorpdfstring{Joint MH - sampling
\(\pmeld(\phi \mid Y_{1}, Y_{2})\)
directly}{Joint MH - sampling \textbackslash{}pmeld(\textbackslash{}phi \textbackslash{}mid Y\_\{1\}, Y\_\{2\}) directly}}\label{joint-mh---sampling-pmeldphi-mid-y_1-y_2-directly}

In this example we can specify the melded posterior in Stan directly.
Figure~\ref{fig:joint_trace} is the traceplot of the sampler that
directly targets the melded posterior.

\begin{figure}

{\centering \includegraphics[width=1\linewidth]{plots/norm-norm-ex/joint/trace} 

}

\caption{Traceplot of the directly sampled melded posterior.}\label{fig:joint_trace}
\end{figure}

We compare the sample quantiles of the melded joint posterior and the
stage two melded posterior obtained using the augmented stage one target
in Figure~\ref{fig:joint_augmented_compare}, and we see good agreement
between the two samples.

\begin{figure}

{\centering \includegraphics[width=1\linewidth]{plots/norm-norm-ex/joint-augmented-compare} 

}

\caption{Quantile-quantile plot of sample quantiles obtained by directly sampling the melded joint posterior (x-axis) and the stage two sample quantiles obtained using the augmented stage one target (y-axis).}\label{fig:joint_augmented_compare}
\end{figure}

\section{Computational costs}\label{computational-costs}

\subsection{Multi-stage Metropolis-Hastings within
Gibbs}\label{multi-stage-metropolis-hastings-within-gibbs}

\subsubsection*{Two submodel case}\label{two-submodel-case}
\addcontentsline{toc}{subsubsection}{Two submodel case}

We would like to generate \(\Nx\) samples from the overall melded
posterior, and hence opt to generate \(\Nx\) samples from the stage one
target. Each stage one sample has a computational cost of order
\(\mathcal{O}(C_{1})\) which involves evaluating the unnormalised
logposterior (\(\mathcal{O}(C_{1, \text{lp}})\)), and generating and
evaluating a proposal for the link parameter
(\(\mathcal{O}(C_{1, \mathcal{Q}})\)). Note that
\(\mathcal{O}(C_{1}) = \mathcal{O}(C_{1, \text{lp}} + C_{1, \mathcal{Q}})\).
The analogous set of costs exist for the second stage target.

First we have to identify that we have a problem:

\begin{itemize}
\tightlist
\item
  The cost of sampling the original stage one target:
  \(\mathcal{O}(\Nx C_{1})\).
\item
  The cost of evaluating said samples in stage two:
  \(\mathcal{O}(\Nx C_{2, \text{lp}})\).
\end{itemize}

Now we can solve it by:

\begin{itemize}
\tightlist
\item
  Sampling the second stage subposterior: \(\mathcal{O}(\Nx C_{2})\).

  \begin{itemize}
  \tightlist
  \item
    This does not include the pooled prior.
  \end{itemize}
\item
  Sampling the augmented stage one target: \(\mathcal{O}(\Nx C_{1}')\).

  \begin{itemize}
  \tightlist
  \item
    The prime indicates that the augmented target has strictly greater
    computational cost to evaluate.
  \end{itemize}
\item
  Evaluating the augmented samples in stage two:
  \(\mathcal{O}(\Nx C_{2, \text{lp}})\).
\end{itemize}

\subsubsection*{More than two submodels}\label{more-than-two-submodels}
\addcontentsline{toc}{subsubsection}{More than two submodels}

If we have \(\Nm > 2\) models, then the costs grow in the following way.

Identifying the problem:

\begin{itemize}
\tightlist
\item
  We finish sampling the \(\modelindex-1\)th stage target:
  \(\mathcal{O}(\Nx (C_{1} + C_{2, \text{lp}} + \ldots + C_{\modelindex - 1, \text{lp}}))\).
\item
  Then, we evaluate these samples under the \(\modelindex\)th stage:
  \(\mathcal{O}(\Nx C_{\modelindex, \text{lp}})\) and realise we have
  conflict.
\end{itemize}

Addressing the issue:

\begin{itemize}
\tightlist
\item
  Sampling all the subposteriors:
  \(\mathcal{O}(\Nx (C_{1} + \ldots + C_{\Nm}))\)
\item
  Sampling the augmented stage one target \(\mathcal{O}(\Nx C_{1}')\).
\item
  Evaluating them under the augmented intermediary targets:
  \(\mathcal{O}(\Nx (C_{2, \text{lp}}' + \ldots + C_{\Nm, \text{lp}}))\).

  \begin{itemize}
  \tightlist
  \item
    Unless we augmented all the intermediary targets, any one stage
    could render the sample degenerate.
  \item
    We do \textbf{not} augment the final target.
  \end{itemize}
\end{itemize}

\subsection{Sequential Monte Carlo
(SMC)}\label{sequential-monte-carlo-smc}

Adopting an SMC approach alleviates the need for initial runs to both
diagnose conflict, and calculate an appropriate augmentation function
\(\tarw_{1}(\phi; \eta_{1})\). Consider the \(\Nm = 2\) models case
initially. If introducing each submodel requires \(T\) tempering steps,
each with a single refreshment step, and we wish to generate \(\Nx\)
samples, the computational costs include:

\begin{itemize}
\tightlist
\item
  Sampling the stage one target: \(\mathcal{O}(\Nx C_{1})\).
\item
  Evaluating the \(T\) tempered likelihoods, and then refreshing the
  particles: \(\mathcal{O}(\Nx T (C_{1} + C_{2}))\). 
\item
  The overall cost is then \(\mathcal{O}(\Nx T (2 C_{1} + C_{2}))\).
\end{itemize}

We can extrapolate this to \(\Nm > 2\) models. Assume introducing each
additional submodel also requires \(T\) tempering steps. The final
computational cost is then
\(\mathcal{O}(\Nx T (\Nm C_{1} + (\Nm - 1) C_{2} + \ldots + C_{\Nm}))\).
Perhaps there are ways to reduce the number of times we need to evaluate
the previous models when introducing the \(\modelindex\)th model. A
careful reading of (Lindsten et al., 2017) seems prudent, as the naive
SMC algorithm I am envisaging here involves evaluating the joint model
(i.e.~all the data).

\section{Example - real world? Where can we get disjoint conflict
from.}\label{example---real-world-where-can-we-get-disjoint-conflict-from.}

\begin{itemize}
\tightlist
\item
  Investigate the Carlin example Rob sent
\end{itemize}

\newpage

\section*{Bibliography}\label{bibliography}
\addcontentsline{toc}{section}{Bibliography}

\hypertarget{refs}{}
\hypertarget{ref-lindsten:etal:17}{}
Lindsten, F., Johansen, A.M., Naesseth, C.A., Kirkpatrick, B., Schön,
T.B., Aston, J.A.D., Bouchard-Côté, A., 2017. Divide-and-Conquer with
Sequential Monte Carlo\emph{. Journal of Computational and Graphical
Statistics} 26, 445--458.
\url{https://doi.org/10.1080/10618600.2016.1237363}

\hypertarget{ref-srivastava:etal:15}{}
Srivastava, S., Cevher, V., Dinh, Q., Dunson, D., 2015. WASP: Scalable
Bayes via barycenters of subset posteriors\emph{, in: Lebanon, G.,
Vishwanathan, S.V.N. (Eds.), Proceedings of the Eighteenth International
Conference on Artificial Intelligence and Statistics, Proceedings of
Machine Learning Research}. PMLR, San Diego, California, USA, pp.
912--920.

\hypertarget{ref-srivastava:li:dunson:18}{}
Srivastava, S., Li, C., Dunson, D.B., 2018. Scalable Bayes via
Barycenter in Wasserstein Space\emph{. J. Mach. Learn. Res.} 19,
312--346.

\newpage

\renewcommand{\thesection}{\Alph{section}}

\setcounter{section}{0}

\section{Appendix}\label{appendix}


\end{document}
